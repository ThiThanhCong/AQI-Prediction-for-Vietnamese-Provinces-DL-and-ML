\documentclass[conference]{IEEEtran}
\IEEEoverridecommandlockouts
% The preceding line is only needed to identify funding in the first footnote. If that is unneeded, please comment it out.
\usepackage{cite}
\usepackage{amsmath,amssymb,amsfonts}
\usepackage{algorithmic}
\usepackage{graphicx}
\usepackage{textcomp}
\usepackage{xcolor}
\usepackage[utf8]{inputenc}
\usepackage[vietnamese]{babel}

\def\BibTeX{{\rm B\kern-.05em{\sc i\kern-.025em b}\kern-.08em
    T\kern-.1667em\lower.7ex\hbox{E}\kern-.125emX}}
\begin{document}

\title{Conference Paper Title*\\
{\footnotesize \textsuperscript{*}Note: Sub-titles are not captured in Xplore and
should not be used}
\thanks{Identify applicable funding agency here. If none, delete this.}
}

\author{\IEEEauthorblockN{1\textsuperscript{st} Cao Hoai Sang}
    \IEEEauthorblockA{\textit{dept. name of organization (of Aff.)} \\
        \textit{name of organization (of Aff.)}\\
        Ho Chi Minh City, Viet Nam \\
        21522541@gm.uit.edu.vn}
    \and
    \IEEEauthorblockN{2\textsuperscript{nd} Nguyen Tran Gia Kiet}
    \IEEEauthorblockA{\textit{dept. name of organization (of Aff.)} \\
        \textit{name of organization (of Aff.)}\\
        City, Country \\
        email address or ORCID}
    \and
    \IEEEauthorblockN{3\textsuperscript{rd} Thi Thành Công}
    \IEEEauthorblockA{\textit{dept. name of organization (of Aff.)} \\
        \textit{name of organization (of Aff.)}\\
        City, Country \\
        email address or ORCID}
    \and
    \IEEEauthorblockN{4\textsuperscript{th} Given Name Surname}
    \IEEEauthorblockA{\textit{dept. name of organization (of Aff.)} \\
        \textit{name of organization (of Aff.)}\\
        City, Country \\
        email address or ORCID}
    \and
    \IEEEauthorblockN{5\textsuperscript{th} Given Name Surname}
    \IEEEauthorblockA{\textit{dept. name of organization (of Aff.)} \\
        \textit{name of organization (of Aff.)}\\
        City, Country \\
        email address or ORCID}
    \and
    \IEEEauthorblockN{6\textsuperscript{th} Given Name Surname}
    \IEEEauthorblockA{\textit{dept. name of organization (of Aff.)} \\
        \textit{name of organization (of Aff.)}\\
        City, Country \\
        email address or ORCID}
}

\maketitle

\begin{abstract}
    This document is a model and instructions for \LaTeX.
    This and the IEEEtran.cls file define the components of your paper [title, text, heads, etc.]. *CRITICAL: Do Not Use Symbols, Special Characters, Footnotes,
    or Math in Paper Title or Abstract.
\end{abstract}

\begin{IEEEkeywords}
    component, formatting, style, styling, insert
\end{IEEEkeywords}

\section{Introduction}
This document is a model and instructions for \LaTeX.
Please observe the conference page limits.

\section{Nghiên cứu liên quan}

\subsection{Gauss-Newton nonlinear method}

Vào năm 2015, ứng dụng dự báo phục hồi sau xuất viện của Phạm Thị Hương được
sử dụng trong luận văn thạc sĩ khoa học [1]. Sử dụng phương pháp Gauss-Newton
phi tuyến để ước lượng giá trị nhỏ nhất của bình phương sai số. \cite{b1}



\section{Tài nguyên}
\section{Phương pháp luận}
\subsection{Gauss newton method nonlinear}
\subsubsection{Least Squares}
Khoảng cách giữa một đường cong được khớp và một quan sát được gọi là sai số dư, hoặc lỗi.

\begin{center}
    $Residuals = y_i - \widehat{y}_i$
\end{center}

Tổng của các sai số bình phương được tính bằng phương trình sau:

\begin{center}
    $SSE = \sum_{i=1}^{n}(y_i - \widehat{y}_i)^2$
\end{center}

Trong đó: \\
\indent\textbullet\ \(y_i\) là các giá trị quan sát được\\
\indent\textbullet\ \(\widehat{y}_i\) là các giá trị khớp\\

\subsubsection{Phương pháp Newton}
Với hàm \( y = y_0 e^{-kt} \), chúng ta tìm giá trị nhỏ nhất của SSE. Chúng ta tìm giá trị \(k\) bằng phương pháp Newton.

\begin{center}
    $SSE = \sum_{i}^{n}(y_i - y_0 e^{-kt_i})^2$
\end{center}

\begin{center}
    \(k_{\text{new}} = k_{\text{old}} - \frac{f'(k_{\text{old}})}{f''(k_{\text{old}})}\)
\end{center}

Hoặc chúng ta có thể giải thích bằng ma trận Hessian như sau:

\begin{center}
    \[
        \begin{pmatrix}
            k_{\text{new}} \\ y_{0,\text{new}}
        \end{pmatrix} =
        \begin{pmatrix}
            k_{\text{old}} \\ y_{0,\text{old}}
        \end{pmatrix} - H^{-1}G
    \]
\end{center}

Trong đó: \\
\indent\textbullet\ \(H = \begin{bmatrix}
    \frac{\partial^2 f}{\partial k^2}            & \frac{\partial^2 f}{\partial k \partial y_0} \\
    \frac{\partial^2 f}{\partial y_0 \partial k} & \frac{\partial^2 f}{\partial y_0^2}
\end{bmatrix}\)\\
\indent\textbullet\ \(G = \begin{bmatrix}
    \frac{\partial f}{\partial k} \\ \frac{\partial f}{\partial y_0}
\end{bmatrix}\)\\

Vấn đề với phương pháp Newton trong hồi quy phi tuyến là việc tính toán ma trận Hessian và nghịch đảo của nó gặp khó khăn. Để giải quyết vấn đề này, phương pháp Gauss-Newton thay thế bằng cách xấp xỉ ma trận Hessian.

Chúng ta có thể viết lại SSE như sau:
\begin{center}
    $SSE = \sum_{i}^{n}r_i^2 = r^T r$
\end{center}

Trong đó: \\
\indent\textbullet\ \(r\) là một vector chứa các sai số dư \\

\subsubsection{Phương pháp Gauss-Newton}
Chúng ta lấy đạo hàm của SSE theo các tham số trong mô hình bằng quy tắc chuỗi, chúng ta có được phương trình sau:
\begin{center}
    \(
    \frac{\partial SSE}{\partial \beta_j} = 2\sum_{i}^{n} r_i \frac{\partial r_i}{\partial \beta_j}
    \)
\end{center}

Sau đó, bỏ số hai vì nó sẽ không ảnh hưởng đến việc ước tính các tham số. Tương ứng với ma trận Jacobian.

\begin{center}
    \(
    J_r = \begin{bmatrix}
        \frac{\partial r_1}{\partial \beta_1} & \frac{\partial r_1}{\partial \beta_2} \\
        \frac{\partial r_2}{\partial \beta_1} & \frac{\partial r_2}{\partial \beta_2} \\
        \vdots                                & \vdots                                \\
        \frac{\partial r_n}{\partial \beta_1} & \frac{\partial r_n}{\partial \beta_2}
    \end{bmatrix}
    \)
\end{center}

Với phương trình sau cho tổng các sai số bình phương (SSE):
\[
    SSE = \sum_{i=1}^{n} (y_i - y_0 e^{-kt_i})^2
\]
Chúng ta có:
\begin{center}
    \(
    \frac{\partial^2 {SSE}}{\partial \beta_j \partial \beta_k} = \sum_{i}^{n} \left( \frac{\partial r_i}{\partial \beta_j} \frac{\partial r_i}{\partial \beta_k} + r_i \frac{\partial^2 r_i}{\partial \beta_j \partial \beta_k} \right)
    \)
\end{center}

Sự khác biệt chính giữa phương pháp Newton và Gauss-Newton là phương pháp Gauss-Newton bỏ qua \(r_i \frac{\partial^2 r_i}{\partial \beta_j \partial \beta_k}\).
Do đó, đạo hàm bậc hai được xấp xỉ bằng hàm sau:
\begin{center}
    \(
    \frac{\partial^2 {SSE}}{\partial \beta_j \partial \beta_k} \approx \sum_{i}^{n} \left( \frac{\partial r_i}{\partial \beta_j} \frac{\partial r_i}{\partial \beta_k} \right) = J^T_r J_r
    \)
\end{center}

Sử dụng quy tắc cập nhật sau trong phương pháp Newton. Đối với Gauss-Newton, đơn giản chỉ cần cắm vào xấp xỉ cho ma trận Hessian và gradient.

\begin{center}
    \[
        \begin{pmatrix}
            k_{\text{new}} \\ y_{0,\text{new}}
        \end{pmatrix} =
        \begin{pmatrix}
            k_{\text{old}} \\ y_{0,\text{old}}
        \end{pmatrix} - (J^T_r J_r)^{-1} J^T_r r
    \]
\end{center}

Với \(\beta\) là một vector cột với các tham số được ước tính. Đối với ví dụ đơn giản chỉ ước tính hai tham số, phương trình trông như sau:

\begin{center}
    \[
        \beta_{\text{new}} = \beta_{\text{old}} - (J^T_r J_r)^{-1} J^T_r r(\beta_{\text{old}})
    \]
    \[
        \begin{pmatrix}
            k_{\text{new}} \\ y_{0,\text{new}}
        \end{pmatrix} =
        \begin{pmatrix}
            k_{\text{old}} \\ y_{0,\text{old}}
        \end{pmatrix} - (J^T_r J_r)^{-1} J^T_r r \begin{pmatrix}
            k_{\text{old}} \\ y_{0,\text{old}}
        \end{pmatrix}
    \]
\end{center}



\begin{thebibliography}{00}
    \bibitem{b1}  P. T. Huong, "Linear regression, polynomial regression, and applications,
    master’s thesis in science," 2015.
\end{thebibliography}

\end{document}
